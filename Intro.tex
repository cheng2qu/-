% !TeX encoding = UTF-8
% !TeX program = LuaLaTeX

\documentclass{tufte-handout}

%%% 设置页面布局
% 设置页边距
\usepackage{geometry}
% 设置缩进
\setlength{\parindent}{0pt}
% 设置段落间距
\setlength{\parskip}{1.5em}
% 设置颜色
\usepackage{color}
% 删除线插件
\usepackage[normalem]{ulem}
% 设置插图
% Set up the images/graphics package
\usepackage{graphicx}

% 设置自定义列表图标
\usepackage{lipsum,graphicx}
% 自定义星星图标
\newcommand{\parStar}{%
    \par\noindent
    \smash{%
    \makebox[0pt]{%
    \raisebox{\dimexpr-.5\height+.3\baselineskip}{%
    \includegraphics[height=7pt]{itemStar.png}%
      }%
    }%
  }%
  \hspace*{\parindent}}

% 自定义星球图标
\newcommand{\parPlant}{%
    \par\noindent
    \smash{%
    \makebox[0pt]{%
    \raisebox{\dimexpr-.5\height+.3\baselineskip}{%
    \includegraphics[height=7pt]{itemPlant.png}%
      }%
    }%
  }%
  \hspace*{\parindent}}
  
% 自定义野花图标
\newcommand{\parFlw}{%
    \par\noindent
    \smash{%
    \makebox[0pt]{%
    \raisebox{\dimexpr-.5\height+.3\baselineskip}{%
    \includegraphics[height=7pt]{itemFlw.png}%
      }%
    }%
  }%
  \hspace*{\parindent}}
%%% CJK language setting

%%% xeCJK package under XeTeX, suitable for Chinese doc
% \usepackage[space]{xeCJK}
%% Chinese as the main language
% \setCJKmainfont{Noto Serif CJK SC}
% \setCJKsansfont{Noto Sans CJK SC}

%% Define fonts for Japanese and Korean
% \newCJKfontfamily\japanesefont{Noto Serif CJK JP}

%%% LuaTexja package under LuaTeX, suitable for Japanese doc
\usepackage{luatexja}            % Basic package
\usepackage{luatexja-ruby}       % 调用振假名插件
\ltjsetruby{size=0.6}            % 设置振假名字号
%\ltjsetruby{fontcmd=\gtfamily}  % 设置振假名字体
\ltjsetruby{mode=00}             % 设置振假名的「進入」和「突出」模式

%\usepackage{pxrubrica}          % 同为振假名插件

\usepackage{luatexja-fontspec}   % 字体设置插件
% 为了实现中日双语混排,需要调用中文字体
\setmainjfont{SIMSUN.TTC}        % 调用嵌入简体宋体

%%% 其他工具包
% 网页超链接
\usepackage{hyperref}

% Spacing setting
\large
\selectfont

% Date in Chinese
\renewcommand{\today}{\number\year 年 \number\month 月 \number\day 日}

%%% Title and author information
\title{「星の王子さま」日语学习笔记}
\author{By-是我 \sout{Dio} Qu啦 \thanks{只是无聊的JOJO梗}}

\begin{document}

\maketitle

\section*{\bf{Hi! 很高兴见到你!}}
这份资料是以《小王子》为基础整理的文型、单字笔记。笔记的排版是之前上课的习题排版,左边是日文,右边是中文翻译、注释和碎碎念。感觉「星の王子さま」的译名还是要比「 あのときの王子くん 」亲切,姑且就先这样吧。如果这一版制作顺利的话,下一本想讲《银河铁道之夜》,广播剧和85版动画都好喜欢\sidenote{这版才开始就在想些有的没的}。

因为现在自己也才刚入门,笔记应该会涉及很多基础知识点。不过五十音拼读还是需要的,有关五十音的入门介绍会整理在附录里。如果出现汉字,笔记里会注上假名。青空这版电子书99\%都是假名,作为学习材料,笔记里也会尽量注上汉字。

\section*{\bf{学习资料传送}}

    \parStar \href{https://www.aozora.gr.jp/cards/001265/files/46817_24670.html}{日版《小王子》电子书}, 收录于青空文库({ 青空文庫})电子图书馆。大量已经公开版权的日文作品都可以在这里找到。
    
    \parStar \href{https://music.163.com/#/album?id=3139032}{「星の王子さま」广播剧},这一辑是在原文基础上分角色演绎的版本。
    
    \parStar 日本老师用中文讲解日语,是目前看到最适合零基础入门的教程=)50p的文法课也没有很枯燥,有配套教程但目前只看视频也还好。小说整理出来的知识点更多只是把联系变得 有意思\sidenote{日剧、日漫、综艺和日文歌都是如此},扎实的语法还是需要系统化的教程呐。
    \begin{itemize}
        \setlength\itemsep{0.5em}
        
        \item \href{https://www.bilibili.com/video/BV1mt411M7Uy/?p=6}{五十音入门-B站} / \href{https://www.youtube.com/playlist?list=PLynCeSdpMqxBipKl9EHnBzZFzBnGuB108}{油管}
        
        \item \href{https://www.bilibili.com/video/BV1NJ41187DK?t=48}{出口日语文法教程-B站} / \href{https://www.youtube.com/playlist?list=PLynCeSdpMqxCW-AfMtmIlASAMUVq8wX6k}{油管}
        
    \end{itemize}
    
    \parStar 在线日文词典工具箱\sidenote{目前还只是跟着视频学语法,这部分之后慢慢补充上来}
    \begin{itemize}
        \setlength\itemsep{0.5em}
        
        \item \href{https://cjjc.weblio.jp/}{Weblio日中中日词典}
        
        \item 移动端上的手写词典应用「白檜辞書」(Shirabe Jisho),适合查不认识的单字发音
    \end{itemize}
    
\newpage

\section*{附录1-五十音}
五十音书写和发音应该是最最基本的要点了吧~
知道汉字起源后,对平片假名印象会更深
我在看视频的时候有反复抄写,之后学单词、文法都会不断加深印象,所以感觉不用特别花时间再去死记硬背
刚开始容易记混时,有用便利贴写上4、5个平片假名,零零星星贴在厕所镜子、窗边、冰箱门这种存在感高的地方,偶尔瞄到提醒一下。



\end{document}
% 格式文件
% !TeX encoding = UTF-8
% !TeX program = LuaLaTeX

\documentclass{tufte-handout}

%%% 设置页面布局
% 设置页边距
\usepackage{geometry}
% 设置缩进
\setlength{\parindent}{0pt}
% 设置段落间距
\setlength{\parskip}{1.5em}
% 设置颜色
\usepackage{color}

% 删除线插件
\usepackage[normalem]{ulem}
% 插图插件
\usepackage{graphicx}

% 设置自定义列表图标
% 自定义星星图标
\newcommand{\parStar}{%
    \par\noindent
    \smash{%
    \makebox[0pt]{%
    \raisebox{\dimexpr-.5\height+.3\baselineskip}{%
    \includegraphics[height=10pt]{itemStar.png}%
      }%
    }%
  }%
  \hspace*{\parindent}}

% 自定义星球图标
\newcommand{\parPlant}{%
    \par\noindent
    \smash{%
    \makebox[0pt]{%
    \raisebox{\dimexpr-.5\height+.3\baselineskip}{%
    \includegraphics[height=10pt]{itemPlant.png}%
      }%
    }%
  }%
  \hspace*{\parindent}}
  
% 自定义野花图标
\newcommand{\parFlw}{%
    \par\noindent
    \smash{%
    \makebox[0pt]{%
    \raisebox{\dimexpr-.5\height+.3\baselineskip}{%
    \includegraphics[height=10pt]{itemFlw.png}%
      }%
    }%
  }%
  \hspace*{\parindent}}
%%% CJK language setting

%%% xeCJK package under XeTeX, suitable for Chinese doc
% \usepackage[space]{xeCJK}
%% Chinese as the main language
% \setCJKmainfont{Noto Serif CJK SC}
% \setCJKsansfont{Noto Sans CJK SC}

%% Define fonts for Japanese and Korean
% \newCJKfontfamily\japanesefont{Noto Serif CJK JP}

%%% LuaTexja package under LuaTeX, suitable for Japanese doc
\usepackage{luatexja}            % Basic package
\usepackage{luatexja-ruby}       % 调用振假名插件
\ltjsetruby{size=0.6}            % 设置振假名字号
%\ltjsetruby{fontcmd=\gtfamily}  % 设置振假名字体
\ltjsetruby{mode=00}             % 设置振假名的「進入」和「突出」模式

%\usepackage{pxrubrica}          % 同为振假名插件

\usepackage{luatexja-fontspec}   % 字体设置插件
% 为了实现中日双语混排,需要调用中文字体
\setmainjfont{SIMSUN.TTC}[AutoFakeBold]        % 调用嵌入的简体宋体

% 修改页眉页脚默认字体
\usepackage{fancyhdr}
\newcommand{\cjkfont}{%
\fontfamily{SIMSUN.TTC}\fontseries{b}\fontsize{9}{11}\selectfont}
% 定义页眉页脚使用的文档部件
\usepackage{titlesec}
\renewcommand{\sectionmark}[1]{%
\markboth{#1}{}}

%%% 其他工具包
% 自定义章节名称插件
\usepackage{titlesec}
% 页码格式
\usepackage[utf8]{inputenc}

% 网页超链接插件
\usepackage{hyperref}
% 超链接高亮颜色设置
% 加载色彩工具包
\usepackage[svgnames]{xcolor}
% 自定义色彩
\definecolor{VolcanoGray}{RGB}{147, 159, 174}
\definecolor{RossRed}{RGB}{146, 71, 62}
\definecolor{Baobab}{RGB}{120, 140, 97}
\definecolor{FoxOrange}{RGB}{239, 171, 93}
\definecolor{ClothGreen}{RGB}{159, 174, 166}
\definecolor{SmokeBlue}{RGB}{79, 129, 155}
% 色彩调用-超链接文本
\hypersetup{
    colorlinks=true,
    linkcolor=SmokeBlue,
    filecolor=RossRed,      
    urlcolor=SmokeBlue,
}
% 色彩调用-页边注(sidenote有编号,marginnote无编号)
\setsidenotefont{\color{VolcanoGray}\footnotesize}
\setmarginnotefont{\color{VolcanoGray}\footnotesize}

% Spacing setting
\large
\selectfont

% Date in Chinese
\renewcommand{\today}{\number\year 年 \number\month 月 \number\day 日}


